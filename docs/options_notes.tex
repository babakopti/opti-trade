\documentclass{article}
\usepackage{graphicx}
\usepackage{amsmath}
\usepackage{amssymb}
\usepackage{natbib}
\renewcommand{\refname}{References}
\usepackage{url}

\title{A Proposed Strategy for Options Trading}

\author{Babak Emami}

\date{\today}

\begin{document}
\maketitle

\section{An Options Trading Stategy}\label{section:options_trading}

Here we propose a strategy for trading of options contracts, based on
the differential geometric forecast model.

When a trader buys a call or put option, the cost includes the option
price and a commission fee. Note that options are often contracts
consisting of 100 units of the underlying security. 

Let us assume that $t_{s}$ is the current time. There is a profit on
exercising a call option $\alpha$ with an underlying security $i$, at
time $t > t_{s}$ when,

\begin{equation}\label{eqn:call-option-condition}
y^{i}(t) > s^{i}_{\alpha} + ( 1 + \xi )^{t-t_{s}} ( \pi^{i}_{\alpha}(t_{s}) + C )
\end{equation}

where $y^{i}(t)$ is the price of the underlying security at time $t$,
$s^{i}_{\alpha}$ is the option strike price, $\pi^{i}_{\alpha}(t_{s})$
is price of the option at $t_{s}$ (per unit in contract, so if there
are 100 unites of the underlying security in the contract, the total
contract price is $100 \pi^{i}_{\alpha}(t_{s})$ ), $C$ is the
commision per unit in the contract, and $\xi$ is the risk free
interest rate per unit of time; note that $i \in [1,\tilde{n}] \cap
\mathbb{N}$. For simplicity of notation, we define $\eta^{c}(t) \equiv
s^{i}_{\alpha} + ( 1 + \xi )^{t-t_{s}} ( \pi^{i}_{\alpha}(t_{s}) + C
)$.

Similarly, a put option is profitable when,

\begin{equation}\label{eqn:put-option-condition}
y^{i}(t) < s^{i}_{\alpha} - ( 1 + \xi )^{t-t_{s}} ( \pi^{i}_{\alpha}(t_{s}) + C )
\end{equation}

For simplicity of notation, we define $\eta^{p}(t) \equiv
s^{i}_{\alpha} - ( 1 + \xi )^{t-t_{s}} ( \pi^{i}_{\alpha}(t_{s}) + C )$.

At a given snapdate $t_{s}$, we build a differential geometric
forecast model with $n$ variables, consisting of $\tilde{n}$
securities, using hirtorical data up to $t_{s}$. We use this model to
devise a strategy to trade on call and put options corresponding to
these securities. The model predicts the price of the undelying
assets, that is $y^{i}(t)$.

Based on the prediction, we should enter a call options contract if
\ref{eqn:call-option-condition} holds. Let us assume $\bar{y}^{i}(t)$
is the expected value of prediction of price of asset $i$, and
$\sigma^{i}$ is the strandard deviation. Assuming a normal
distribution, the probability of the condition
\ref{eqn:call-option-condition} to hold is,

\begin{equation}\label{eqn:call-option-prob-intg}
Pr[ y^{i}(t) > \eta^{c}(t) ] = \int_{\eta^{c}(t)}^{\infty}
\frac{1}{\sqrt{2\pi\sigma_{i}^{2}}}
e^{\frac{-(\zeta-\bar{y}^{i}(t))^{2}}{2\sigma_{i}^{2}}} d\zeta
\end{equation}

This yields,

\begin{equation}\label{eqn:call-option-prob}
Pr[ y^{i}(t) > \eta^{c}(t) ] = \frac{1}{2} [ 1 - erf(
  \frac{\eta^{c}(t)-\bar{y}^{i}(t)}{\sqrt{2} \sigma_{i}} ) ]
\end{equation}

Similary for a put option, the probability of
condition \ref{eqn:put-option-condition} to hold is

\begin{equation}\label{eqn:put-option-prob-intg}
Pr[ y^{i}(t) < \eta^{p}(t) ] =
\int_{-\infty}^{\eta^{p}(t)} \frac{1}{\sqrt{2\pi\sigma_{i}^{2}}}
e^{\frac{-(\zeta-\bar{y}^{i}(t))^{2}}{2\sigma_{i}^{2}}} d\zeta
\end{equation}

This yields,

\begin{equation}\label{eqn:put-option-prob}
Pr[ y^{i}(t) < \eta^{p}(t) ] = \frac{1}{2} [ erf(
  \frac{\eta^{p}(t)-\bar{y}^{i}(t)}{\sqrt{2} \sigma_{i}} ) -
  erf( \frac{\bar{y}^{i}(t)}{\sqrt{2} \sigma_{i}} ) ]
\end{equation}

The strategy consists of the following steps,

\begin{itemize}
  
  \item[1] Loop through all assets in the pool and get the options
    chain for each asset. We can limit the expiration dates. For
    instance, we can get the options that mature no sooner than a week
    and no later than three months from now. We can further limit the
    options chain by filtering out contracts that have not been traded
    lately, for instance in the last business day. Furthermore, given
    our capital we can limit what we are willing to spend on one
    contract and filter out contracts accordingly. For example, we can
    limit what we pay for each contract to \$4,000, and assuming the
    the contract has 100 units of underlying security, the unit price
    cannot be above \$200. We then collect all options for all assets
    into one big list. Note that when working with popular assets such
    as SPY, this list can consist of several thousand contracts. Note
    too that this list has both call and put options.

  \item[2] Loop through all contracts and calculate the probability of
    thet contract being profitable using
    Eq.~\ref{eqn:call-option-prob} for call options and
    Eq.~\ref{eqn:put-option-prob} for put options. Sort the list of
    contracts based on the corresponding probabilities. Note that
    effectively this probability represents the projected return to
    risk ratio on a contract (see Eqs.~\ref{eqn:call-option-prob} and
    \ref{eqn:put-option-prob}).

   \item[3] Enter into a certain number of top contracts in the list,
     for instance the first 10.

\end{itemize}

If we already have some options in our portfolio, we can settle each
existing contract by one of three following actions, 

\begin{itemize}

   \item[1] Sell the contract. The expected return on unit underlying
     asset will be $\pi^{i}_{\alpha}(t_{curr}) -
     \pi^{i}_{\alpha}(t_{s}) - C$, where $t_{curr}$ is the current
     time. Th eexpected return can be negative.

   \item[2] Exercise the contract. The expected return per unit of
     underlying asset will be $y^{i}(t_{curr}) - \eta^{c}(t_{curr})$
     for call options and $\eta^{p}(t_{curr}) - y^{i}(t_{curr})$ for
     put options.

   \item[3] Not take any action. A reasonable expected return for this
     scenario is $Pr[ y^{i}(t_{m}) > \eta^{c}(t_{m}) ] (y^{i}(t_{m}) -
     \eta^{c}(t_{m}))$ for call options and $Pr[ y^{i}(t_{m}) <
       \eta^{p}(t_{m}) ] (\eta^{p}(t_{m}) - y^{i}(t_{m}))$ for put
     options, where the probilities are calculated from
     Eqs.~\ref{eqn:call-option-prob} and \ref{eqn:put-option-prob},
     and $t_{m}$ is the maturity date.

\end{itemize}

\end{document}
