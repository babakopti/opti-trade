\documentclass{article}
\usepackage{graphicx}
\usepackage{amsmath}
\usepackage{amssymb}
\usepackage{natbib}
\renewcommand{\refname}{References}
\usepackage{url}

\title{A Proposed Strategy for Options Trading}

\author{Babak Emami}

\date{\today}

\begin{document}
\maketitle

\section{An Options Trading Stategy}\label{section:options_trading}

Here we propose a strategy for trading of options contracts, based on
the differential geometric forecast model.

When a trader buys a call or put option, the cost includes the option
price and a commission fee. Note that options are often contracts
consisting of 100 units of the underlying security. 

Let us assume that $t_{s}$ is the current time. There is a profit on
exercising a call option at time $t_{e} > t_{s}$ when,

\begin{equation}\label{eqn:call-option-condition}
y^{i}(t_{e}) > s_{e}^{i} + ( 1 + \xi )^{t_{e}-t_{s}} ( \pi^{i} + C )
\end{equation}

where $y^{i}(t)$ is the price of the underlying security at time $t$,
$s^{e}_{i}$ is the strike price of the options, $\pi^{i}$ is price of
the options (per unit in options contract), $C$ is the commision per
unit in the options contract, and $\xi$ is the risk free interest rate
per unit of time; note that $i \in [1,\tilde{n}] \cap \mathbb{N}$. For
simplicity of notation, we define $\eta^{c}(t_{e}) \equiv s_{e}^{i} +
( 1 + \xi )^{t_{e}-t_{s}} ( \pi^{i} + C )$.

Similaryly, a put option is profitable when,

\begin{equation}\label{eqn:put-option-condition}
y^{i}(t_{e}) < s_{e}^{i} - ( 1 + \xi )^{t_{e}-t_{s}} ( \pi^{i} + C )
\end{equation}

For simplicity of notation, we define $\eta^{p}(t_{e}) \equiv
s_{e}^{i} - ( 1 + \xi )^{t_{e}-t_{s}} ( \pi^{i} + C )$.

At given snapdate $t^{s}$, we build a differential geometric forecast
model with $n$ variables, consisting of $\tilde{n}$ securities. We use
this model to devise a strategy to trade on call and put options
corresponding to these securities. The model predicts the price of the
undelying assets that is $y^{i}(t)$.

Based on the prediction, we should enter a call options contract if
\ref{eqn:call-option-condition} holds. Let us assume $\bar{y}^{i}(t)$
is the expected value of prediction of price of asset $i$, and
$\sigma^{i}$ is the strandard deviation. Assuming a normal
distribution, the probability of the condition
\ref{eqn:call-option-condition} to hold is,

\begin{equation}\label{eqn:call-option-prob-intg}
Pr[ y^{i}(t_{e}) > \eta^{c}(t_{e}) ] = \int_{\eta^{c}(t_{e})}^{\infty}
\frac{1}{\sqrt{2\pi\sigma_{i}^{2}}}
e^{\frac{-(\zeta-\bar{y}^{i}(t_{e}))^{2}}{2\sigma_{i}^{2}}} d\zeta
\end{equation}

This yields,

\begin{equation}\label{eqn:call-option-prob}
Pr[ y^{i}(t_{e}) > \eta^{c}(t_{e}) ] = \frac{1}{2} [ 1 - erf(
  \frac{\eta^{c}(t_{e})-\bar{y}^{i}(t_{e})}{\sqrt{2} \sigma_{i}} ) ]
\end{equation}

Similary for a put option, the probability of
condition \ref{eqn:put-option-condition} to hold is

\begin{equation}\label{eqn:put-option-prob-intg}
Pr[ y^{i}(t_{e}) < \eta^{p}(t_{e}) ] =
\int_{-\infty}^{\eta^{p}(t_{e})} \frac{1}{\sqrt{2\pi\sigma_{i}^{2}}}
e^{\frac{-(\zeta-\bar{y}^{i}(t_{e}))^{2}}{2\sigma_{i}^{2}}} d\zeta
\end{equation}

This yields,

\begin{equation}\label{eqn:call-option-prob}
Pr[ y^{i}(t_{e}) < \eta^{p}(t_{e}) ] = \frac{1}{2} [ erf(
  \frac{\eta^{p}(t_{e})-\bar{y}^{i}(t_{e})}{\sqrt{2} \sigma_{i}} ) -
  erf( \frac{\bar{y}^{i}(t_{e})}{\sqrt{2} \sigma_{i}} ) ]
\end{equation}




\end{document}
