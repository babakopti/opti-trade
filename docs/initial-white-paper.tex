\documentclass{article}

\title{A Differential Geometric Approach to Economic Forecast}

\author{Babak Emami}

\date{\today}

\begin{document}
\maketitle

\begin{abstract}

An approach is proposed to forecast of economic variables. We look at
the economy as multi-demensional manifold wher each dimension is an
economic variable. These can be marco-economic indicators or asset
prices. We refer to this manifold as an economic universe. The
observed values of economic variables in a time period form a ``path''
in this universe. We refer to this path as an economic path. We then
find a manifold for which this path is a geodesic, governed by the
geodesic differential equation. The fututre value of economc variables can be predicted by
solving this difrential equation
.
\end{abstract}

\section{Introduction}\label{introduction}


\paragraph{Outline}
The remainder of this article is organized as follows.
Section~\ref{previous work} gives account of previous work.
Our new and exciting results are described in Section~\ref{results}.
Finally, Section~\ref{conclusions} gives the conclusions.

\section{Previous work}\label{previous work}
A much longer \LaTeXe{} example was written by Gil~\cite{Gil:02}.

\section{Results}\label{results}
In this section we describe the results.

\section{Conclusions}\label{conclusions}
We worked hard, and achieved very little.

\bibliographystyle{abbrv}
\bibliography{main}

\end{document}

