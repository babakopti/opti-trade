\documentclass{article}
\usepackage{graphicx}
\usepackage{amsmath}
\usepackage{amssymb}
\usepackage{natbib}

\title{A Momentum Asset Allocation Strategy}

\author{Babak Emami}

\begin{document}

\maketitle

\section{Introduction}\label{section:introduction}

In what follows, a momentum based asset allocation strategy is
proposed. The idea is to maximize portfolio return, while the
associated portfolio risk, presented by the variance of the portfolio
return, is kept within a given limit.

\section{Mathematical Framework}
\label{section:framework}

The return of any portfolio is calculated as,

\begin{equation}\label{eqn:port-return}
  r = \sum_{i=1}^{K} w_i r_i
\end{equation}

where the portfolio consists of $K$ assets, $w_i$ and $r_i$ are the
weight and return of asset $i \in \{1,...,M\}$, respectively.

The optimization problem is formulated as,

\begin{equation}\label{eqn:objective-func-1}
  \min_{{\bf{w}}} f({\bf{w}}) 
\end{equation}

where the objective function $f$ is defined as,

\begin{equation}\label{eqn:objective-func-1}
  f({\bf{w}}) = -E[r]
\end{equation}

subject to

\begin{equation}\label{eqn:constraints-1}
  VAR(r) \le \alpha^2 VAR(r_{SP})
\end{equation}

where $r_{SP}$ is the return of S\&P500 index or any other index and
$\alpha$ is a model hyper-parameter. Note that the above constraint
ensures that risk associated with portfolio is constrained. Choosing
an $\alpha < 1$ ensures that the portfolio is more conservative than
S\&P500 index, whereas an $\alpha > 1$ indicates that a higher risk
is tolerated in order to get a higher return.

We compute the expection $E[r]$ and variance $VAR(r)$ of portfolio
over a period of time using $M$ overlapping sub-windows. This enures that
minimzation is not affected by a unique financial event that happend
in that period of time. The optimization problem then reduces to,

\begin{equation}\label{eqn:objective-func-2}
  \min_{{\bf{w}}} -\frac{1}{M} \sum_{m=1}^{M} \sum_{i=1}^{K} w_i \bar{r_i}^{(m)}
\end{equation}

subject to $M$ constraints

\begin{equation}\label{eqn:constraints-2}
  C^{(m)}({\bf{w}}) \ge 0
\end{equation}

for all $m \in \{1,...,M\}$, where

\begin{equation}\label{eqn:constraints-def}
  C^{(m)}({\bf{w}}) = \alpha^2 VAR(r_{SP}^{(m)}) - \sum_{i=1}^{K} \sum_{j=1}^{K} w_i w_j COV(r_i^{(m)}, r_j^{(m)})
\end{equation}

Note that $r_{SP}^{(m)}$ is the return of S\&P 500 index (or any other
index of choice) computed on sub-window $m$, and $r_i^{(m)}$ is the
return of asset $i$ calculated on sub-window $m$; $m \in \{1,...,M\}$.

The gradient of the objective function is calculated as,

\begin{equation}\label{eqn:obj-gradient}
  \frac{\partial f}{\partial w_l} = -\frac{1}{M} \sum_{m=1}^{M} \bar{r_l}^{(m)}
\end{equation}

and the graidient of each constraint function $C^{(m)}$ is calculated as,

\begin{equation}\label{eqn:constraints-gradient}
  \frac{\partial C^{(m)}}{\partial w_l} = -2 \sum_{i=1}^{K} w_i COV(r_l^{(m)}, r_i^{(m)})
\end{equation}

\section{Results}\label{section:results}


\end{document}

